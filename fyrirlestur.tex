\documentclass[12pt]{article}
\usepackage[icelandic,english]{babel}
\usepackage{a4,t1enc,html,url}
\usepackage{ucs}
\usepackage{times}
\usepackage[utf8x]{inputenc}
%\pagestyle{empty}
\usepackage{graphicx,latexsym}
\parindent 0pt
%
\usepackage{amsmath} 
\usepackage{graphicx,latexsym} 
\usepackage{amssymb}
\author{Máni Maríus Viðarsson, Jón Blöndal}
%----------------------------------------------------------



\title{p-Legar tölur}
\begin{document}
\maketitle
\section*{Fyrstu kynni - dæmi}
%Hér setjum við sýnidæmi til að vekja áhuga.
Í lok þessara tveggja fyrirlestra mun ykkur vonandi ekki brenna í augun á að sjá eftirfarandi jöfnu:
\begin{equation*}
-1 = 4 + 4\cdot 5+4\cdot 5^2+4\cdot 5^3 \ldots
\end{equation*}
eða 
\begin{equation*}
\sqrt{7} = 1 + 1\cdot 3 + 1\cdot 3^2+0\cdot 3^3+2\cdot 3^4 \ldots
\end{equation*}

%Hér kemur fyrri fyrirlestur
\section*{Inngangur}
Festum prímtölu p og skoðum formlegu veldaröðina
\begin{align*}
a = a_0 + a_1 p + a_2 p^2 + ...,
\end{align*}
þar sem $0 \leq a_i \leq p-1$. 
Svona veldaraðir er hægt að leggja saman, draga frá og margfalda.
Með þessum hætti fáum við bauginn $\mathbb{Z}_p$, p-legu heiltölurnar.
Þær hafa núllstakið $0 = 0+0p+0p^2+...$ og 
einingarstakið $1 = 1 + 0p+0p^2+...$.
\\Baugurinn $\mathbb{Z}_p$ hefur enga núlldeila svo hann er heilbaugur 
og hefur brotasviðið $\mathbb{Q}_p$ sem kallað er svið p-legu talnanna.

%Hér viljum við skilgreina allt reglubundið
\section*{Skilgreiningar}
Ef við skerum af formlegu veldaröðinni við einhvert veldið á p fæst heil tala 
\begin{align*}
a = a_0 + a_1 p + a_2 p^2 + \ldots + a_{n-1}p^{n-1}. 
\end{align*}
Munum að í baugnun $\mathbb{Z}_p$ má geyma og fá lánað í samlagningu, ef við 
fáum $1$ lánaðan frá $p^n$ og skerum af við veldið $n$, fáum við 
\begin{align*}
 a_0 + a_1 p + a_2 p^2 + \ldots + (a_{n-1} + p)p^{n-1}.
\end{align*}
Þessi afskurður verður aðeins vel skilgreindur ef við túlkum hvert 
stak eftir afskurðinn sem stak í $\mathbb{Z}/p^n \mathbb{Z}$, þ.e. það 
er í lagi að geyma og fá lánað.\\
\\Skoðum nú vörpunina
\begin{align*}
 \centering \mathbb{Z}/p^{n+1} \mathbb{Z} \rightarrow 
 \mathbb{Z}/p^{n} \mathbb{Z}, \ \  \left[\sum_{j=0}^k a_j p^j \right]
_{p^{k+1}} \mapsto \left[\sum_{j=0}^{k-1} a_j p^j \right]_{p^k}.
\end{align*}



%Hér byrjar svo seinni fyrirlestur
%BLS 47 í Mahler
\section*{Samfelld föll}
Látum $I=[Q_p]$ tákna mengi allra p-legra heiltalna
\begin{equation*}
x = a_0 + a_1p+a_2p^2 + \ldots \qquad (a_k \mbox{digits}), 
\end{equation*} 
þ.e.a.s. mengi p-legra talna sem uppfylla
\begin{equation*}
 |x|_p \leq 1 .
\end{equation*}
%Setja skilgreiningar haus hér.
Við skilgreinum föll
%Laga þessa setningu eitthvað. 
\begin{equation*}
f:I \rightarrow Q_p
\end{equation*}
samfelld ef fyrir 
\begin{equation*}
x_0 = \lim_{n \rightarrow \infty} x_n
\end{equation*}
gildir
\begin{equation*}
f(x_0) = \lim_{n \rightarrow \infty} f(x_n).
\end{equation*}
Jafngilt er að segja að fall er samfellt e.f.f. fyrir sérhverja heiltölu $s\geq 0$ er til heiltala $t = t(x_0,s)\geq 0$ þ.a.
\begin{equation*}
\mbox{ef } |x-x_0|_p \leq p^{-t},\qquad \mbox{  þá } |f(x)-f(x_0)|_p \leq p^{-s}.
\end{equation*}
Einnig má yfirfæra skilgreininguna á almennari bil
\begin{equation*}
I^* : |x-a|_p \leq p^{-r}
\end{equation*} 
þar sem $a \in I$ er fast og r er fest ræð tala; en þá gefur línulega vörpunin 
\begin{equation*}
x \rightarrow p^rx+a
\end{equation*}
fyrri skilgreiningu.
%Setning
\begin{setn}
Ef $f(x)$ er samfellt á $I$, þá er $f(x)$ samfellt í jöfnum mæli á I, þ.e.a.s fyrir hvert $s$, er til $t=t(s)$ óháð $x_0$ þ.a 
\begin{equation}
\mbox{ef   } x,x_0\in I \mbox{ og  } |x-x_0|_p \leq p^{-t}, \qquad \mbox{þá  } |f(x)-f(x_0)|_p \leq p^{-s}.
\end{equation}
\end{setn} 
\end{document}
