\documentclass[12pt]{article}
\usepackage[icelandic,english]{babel}
\usepackage{a4,t1enc,html,url}
\usepackage{ucs}
\usepackage{times}
\usepackage[utf8x]{inputenc}
%\pagestyle{empty}
\usepackage{graphicx,latexsym}
\parindent 0pt
%
\usepackage{amsmath} 
\usepackage{graphicx,latexsym} 
\usepackage{amssymb}
\author{Máni Maríus Viðarsson, Jón Blöndal}
%----------------------------------------------------------



\title{p-Legar tölur}
\begin{document}
\maketitle
\section*{Fyrstu kynni - dæmi}
%Hér setjum við sýnidæmi til að vekja áhuga.
Í lok þessara tveggja fyrirlestra mun ykkur vonandi ekki brenna í augun á að sjá eftirfarandi jöfnu:
\begin{equation*}
-1 = 4 + 4\cdot 5+4\cdot 5^2+4\cdot 5^3 \ldots
\end{equation*}
eða 
\begin{equation*}
\sqrt{7} = 1 + 1\cdot 3 + 1\cdot 3^2+0\cdot 3^3+2\cdot 3^4 \ldots
\end{equation*}

%Hér kemur fyrri fyrirlestur
\section*{Inngangur}
Festum prímtölu p og skoðum formlegu veldaröðina
\begin{align*}
a = a_0 + a_1 p + a_2 p^2 + ...,
\end{align*}
þar sem $0 \leq a_i \leq p-1$. 
Svona veldaraðir er hægt að leggja saman, draga frá hvorri annarri
 og margfalda.
Með þessum hætti fáum við bauginn $\mathbb{Z}_p$, p-legu heiltölurnar.
Þær hafa núllstakið $0 = 0+0p+0p^2+...$ og 
einingarstakið $1 = 1 + 0p+0p^2+...$.
\\Baugurinn $\mathbb{Z}_p$ hefur enga núlldeila svo hann er heilbaugur 
og hefur brotasviðið $\mathbb{Q}_p$ sem kallað er svið p-legu talnanna.

%Hér viljum við skilgreina allt reglubundið
\section*{Skilgreiningar}
Ef við skerum af formlegu veldaröðinni við einhvert veldið á p fæst heil tala 
\begin{align*}
a = a_0 + a_1 p + a_2 p^2 + \ldots + a_{n-1}p^{n-1}. 
\end{align*}
Munum að í baugnun $\mathbb{Z}_p$ má geyma og fá lánað í samlagningu, ef við 
fáum $1$ lánaðan frá $p^n$ og skerum af við veldið $n$, fáum við 
\begin{align*}
 a_0 + a_1 p + a_2 p^2 + \ldots + (a_{n-1} + p)p^{n-1}.
\end{align*}
Þessi afskurður verður aðeins vel skilgreindur ef við túlkum hvert 
stak eftir afskurðinn sem stak í $\mathbb{Z}/p^n \mathbb{Z}$, þ.e. það 
er í lagi að geyma og fá lánað.\\
\\Skoðum nú vörpunina
\begin{align*}
 \centering \phi_n : \mathbb{Z}/p^{n+1} \mathbb{Z} \rightarrow 
 \mathbb{Z}/p^{n} \mathbb{Z}, \ \  \left[\sum_{j=0}^k a_j p^j \right]
_{p^{k+1}} \mapsto \left[\sum_{j=0}^{k-1} a_j p^j \right]_{p^k}.
\end{align*}
Við fáum
\begin{align}\label{Zruna}
 \ldots \rightarrow \mathbb{Z}/p^{3} \mathbb{Z} \rightarrow \mathbb{Z}/p^{2} 
\mathbb{Z} \rightarrow \mathbb{Z}/p \mathbb{Z}
\end{align}
\begin{skilgr}
 Baugur p-legu heiltalnanna $\mathbb{Z}_p$ er öfuga markgildið af 
röðinni $\ref{Zruna}$
\end{skilgr}
Stak $x$ í $\mathbb{Z}_p$ er runan $(\ldots , x_n, \ldots, x_1)$ þar sem 
$x_n \in \mathbb{Z}/p^n\mathbb{Z}$ og $\phi_n(x_{n+1}) = x_{n}$.
Athugum nú að $\mathbb{Z}_p$ er hlutbaugur í mengjamargfeldinu
 $\prod_{n \geq 1} \mathbb{Z}/p^n\mathbb{Z}$. Ef við setjum á 
$\mathbb{Z}/p^n\mathbb{Z}$ strjála grannmynstrið og á 
$\prod_{n \geq 1} \mathbb{Z}/p^n\mathbb{Z}$ 
margfeldisgrannmynstrið, erfir $\mathbb{Z}_p$ grannmynstrið sem 
gerir það að þjöppuðu rúmi.\\
%\\Látum nú $\varepsilon_n : \mathbb{Z}_p \rightarrow \mathbb{Z}/p^n 
%\mathbb{Z}$ vera fallið sem varpar p-legri heiltölu $x$ á n-ta stakið $x_n$.
%\begin{setn}
% Lestin $0 \rightarrow \mathbb{Z}_p \rightarrow^{p^n} \mathbb{Z}_p 
%\rightarrow^{\varepsilon_n} \mathbb{Z}/p^n\mathbb{Z} \rightarrow 0$ 
%er fleyguð.
%\end{setn}
%(Því getum við greint $\mathbb{Z}_p/p^n\mathbb{Z}$ með 
%$\mathbb{Z}/p^n\mathbb{Z}$).
% \begin{proof}
% Margföldun með $p$ (og því einnig með $p^n$) er eintæk í 
% $\mathbb{Z}_p$; ef $x = (x_n)$ er p-leg heiltala þ.a. $px = 0$,
% fáum við $px_{n+1} = 0$ fyrir öll $n$, og $x_{n+1}$ er á forminu 
% $p^ny_{n+1}$ með $y_{n+1} \in \mathbb{Z}_p/p^{n+1}\mathbb{Z}$; 
% fyrst $x_n = \phi_{n+1}(x_{n+1})$, sjáum við að $x_n$ er deilanleg með 
% $p^n$ svo það er núll.\\
% Það er skýrt að $p^n \mathbb{Z}_p \subset Ker(\varepsilon_n)$; öfugt ef 
% $x = (x_m)$ er í 
% \end{proof}
\begin{setn}
 (a) Til að stak í $\mathbb{Z}_p$ hafi umhverfu er nauðsynlegt og nægjanlegt 
að það sé ekki deilanlegt með $p$.\\
 (b) Ef $\mathbb{U}$ er mengi umhverfanlegra staka í $\mathbb{Z}_p$ er hægt að 
skrifa hvert stak (nema núll) á óvtvírætt ákvarðaðan hátt á forminu $p^nu$ þar
 sem $u \in \mathbb{U}$ og $n\geq 0$.
\end{setn}
Látum nú $\varepsilon_n : \mathbb{Z}_p \rightarrow \mathbb{Z}/p^n 
\mathbb{Z}$ vera fallið sem varpar p-legri heiltölu $x$ á n-ta stakið $x_n$.
\begin{proof}
 Nóg að sanna (a) fyrir $\mathbb{Z}/p^n\mathbb{Z}$, tilfellið fyrir 
$\mathbb{Z}_p$ fylgir. Ef $x \in \mathbb{Z}/p^n\mathbb{Z}$ tilheyrir 
ekki $p\mathbb{Z}/p^n\mathbb{Z}$, þá er mynd þess í 
$\mathbb{Z}/p\mathbb{Z}$ ekki núll, og því umhverfanlegt: 
því eru til $y,z \in \mathbb{Z}/p^n\mathbb{Z}$ þannig að 
$xy = 1-pz$, og því 
\begin{align*}
 xy(1+pz+\ldots+p^{n-1}z^{n-1}) = 1,
\end{align*}
sem sannar að x er umhverfanlegt.\\
Hinsvegar, ef $x \in \mathbb{Z}_p$ er ekki núll, þá er til stærsta heiltala 
$n$ þannig að $x_n = \varepsilon_n(x)$ er núll; þá er $x = p^nu$ með $u$ ekki 
deilanlegt með p, og því er $u \in \mathbb{U}$ vegna (a). 
\end{proof}
\begin{skilgr}
 Virðing á sviði K er vörpun $K \rightarrow \mathbb{R}$ þ.a.
\begin{align*}
 i)& \  |x|\geq 0; |x| = 0 \mbox{ þþaa } x = 0 \\
 ii)& \  |xy| = |x||y| \\
 iii)& \  |x+y| \leq |x| + |y|
\end{align*}
\end{skilgr}
\begin{Rit}
Látum $x \in \mathbb{Z}_p$, $x \neq 0$; skrifum $x$ á forminu $p^nu$ með 
$u \in \mathbb{U}$. Talan $n$ er kölluð $p$-lega virðingin á $x$ og er skrifuð sem 
$v_p(x)$. Við setjum $v_p(0) = +\infty$ og höfum 
\begin{align*}
 v_p(xy) &= v_p(x) + v_p(y) \\
 v_p(x+y) &\geq \inf(v_p(x),v_p(y))
\end{align*}
Augljóst er af þessum formúlum að $\mathbb{Z}_p$ er heilbaugur.

\end{Rit}
\begin{setn}
 Grannmynstrið á $\mathbb{Z}_p$ má skilgreina með firðinni
\begin{align}
 d(x,y) = e^{-v_p(x-y)}
\end{align}
Baugurinn $\mathbb{Z}_p$ er fullkomið firðrúm og $\mathbb{Z}$  er þétt 
í honum.
\end{setn}
\begin{proof}
 Íðulin $p^n \mathbb{Z}_p$ mynda grunn af grenndum um 0 og vegna þess að 
$x \in p^n \mathbb{Z}_p$ er jafngilt því að $v_p(x) \geq n$ er grannmynstrið 
á $\mathbb{Z}_p$ skilgreint með firðinni $d(x,y) = e^{-v_p(x-y)}$. 
Fyrst að $\mathbb{Z}_p$ er þjappað, er það fullkomið. Að lokum, ef $x = (x_n)$
 er stak í $\mathbb{Z}_p$ og ef $y_n \in \mathbb{Z}$ þannig að 
$y_n \equiv x_n(mod p^n)$, þá er $\lim y_n = x$, sem sannar að $\mathbb{Z}$ 
er þétt í $\mathbb{Z}_p$.
\end{proof}

\begin{skilgr}
 Svið $p$-legu talnanna $\mathbb{Q}_p$ er brotasvið baugsins $\mathbb{Z}_p$.
\end{skilgr}
Við sjáum strax að $\mathbb{Q}_p = \mathbb{Z}_p[p^{-1}]$. Hvert stak 
$x \in \mathbb{Q}_p^*$ er hægt að skrifa á ótvírætt ákvarðaðan hátt á forminu 
$p^nu$ með $n \in \mathbb{Z}$, $u \in \mathbb{U}$; hér er $n$ aftur $p$-lega 
virðingin af $x$ og er skrifað sem $v_p(x)$. Við höfum $v_p(x) \geq 0$ þþaa 
$x \in \mathbb{Z}_p$. 
\begin{setn}
 Sviðið $\mathbb{Q}_p$, með grannmynstrinu skilgreindu af 
$d(x,y) = e^{-v_p(x-y)}$, er staðþjappað og inniheldur $\mathbb{Z}_p$ sem 
hlutbaug; sviðið $\mathbb{Q}$ er þétt í $\mathbb{Q}_p$.
\end{setn}
\begin{proof}
 Augljóst.
\end{proof}
$Athugasemdir$\\
(1) Við hefðum getað skilgreint $\mathbb{Q}_p$ sem fullkomnuna á $\mathbb{Q}$ 
fyrir $p$-legu firðina $d$.\\
(2) Firðin $d$ er ofurfirð, það er 
\begin{align*}
 d(x,y) \leq \sup (d(x,y),d(y,z)).
\end{align*}
Af þessu sjáum við að runa $u_n$ hefur markgildi þþaa 
\begin{align*}
 \lim(u_{n+1} - u_n) = 0;
\end{align*}
og á sama hátt, röð er samleitin þþaa runa hennar stefni á 0.

%Hér byrjar svo seinni fyrirlestur

%bls 25 Bachmann
\section*{Samleitnar runur}
\begin{setn}
Ef $k$ er fullkomið svið m.t.t. óarkemedískar virðingar $\| \cdot  \|$, og ef $\{ a_n \} $ er 
runa af stökum í $k$ þ.a. $\lim a_n = 0$, þá er $\sum_{n=1}^\infty a_n$ samleitin.
\end{setn}
\begin{proof}
Lát 
\begin{equation*}
s_n = a_1+ \ldots + a_n
\end{equation*}
og
\begin{equation*}
s_m = a_1 + \ldots + a_m,
\end{equation*}
þar sem $m < n$. Þá
\begin{equation*}
\|s_n - s_m \| = \| a_n + \ldots + a_{m+1} \| \leq \max_{m+1 \leq i \leq n} \|a_i\| \rightarrow 0
\end{equation*}
þegar $n,m \rightarrow \infty$, því er $\lim s_n$ til þar sem $k$ er fullkomið. Þ.a.l er $\sum_{n=1}^\infty a_n$ samleitin.
\end{proof}

\begin{setn}
Ef $k$ er fullkomið svið m.t.t. virðingar $\| \cdot \|$, og ef $\{ a_n \}$ er runa af stökum í $k$ 
þ.a. $\sum_{n=1}^{\infty} \| a_n \| $ er samleitin, þá er $\sum_{n=1}^\infty a_n $ samleitin. 
\end{setn}
\begin{proof}
Lát 
\begin{equation*}
 s_n' = \| a_1 \| + \ldots + \|a_n\|, \qquad s_n = a_1 + \ldots + a_n.
\end{equation*}
Þá fyrir $n>m$, 
\begin{equation*}
|s_n'- s_m' | = | \| a_n \| + \ldots \|a_{m+1}\| |.
\end{equation*}
þar sem algildið $| s_n' - s_m' | $ vísir til venjulega  tölugildisins á $\mathbb{R}$. En 
\begin{equation*}
| \| a_n \| + \ldots + \| a_{m+1} \| | = \| a_n \| + \ldots + \| a_{m+1} \|
\end{equation*}
því $\| a_i \| \leq 0$, og 
\begin{equation}
\| s_n - s_m \| = \|a_n\| + \ldots + \|a_{m+1}\| \rightarrow 0
\end{equation}
þegar $n,m \rightarrow \infty$ fyrst að $\sum_{n=1}^\infty \|a_n\|$ er samleitin. 
Því fæst $\|s_n - s_m \| \rightarrow 0$ þegar $n,m \rightarrow \infty $ og þar sem $k$ er fullkomið fæst niðurstaðan.  
\end{proof}

Fyrir veldaröð $\sum_{n=0}^{\infty} a_nx^n$ í $\mathbb{R}$ eða $\mathbb{C}$ segjum við að hún hafi samleitnigeislan $r$ 
m.t.t. p-lega staðalsins ef fyrir öll x þ.a. $\|x\|_p < r$ veldaröðin er samleitin, en ósamleitin fyrir öll x þ.a. $\|x\|_p > r$, 
þar sem r er gefið með
\begin{equation*}
r = \frac{1}{\limsup \sqrt{\|a_n\|_p }}.
\end{equation*} 

%Hér vantar örugglega smá
%Rökstyðja þetta
Við höfum að 
\begin{equation}
\label{starwars}
\mbox{ord}_p xy = \mbox{ord}_p x + \mbox{ord}_p y
\end{equation}
sem leiðir beint af $\|xy\|_p = \|x\|_p \|y\|_p$.
%----
\begin{hjalparsetn}
Lát $n$ vera jákvæða heiltölu og 
\begin{equation*}
n = a_0 + a_1p+ \ldots + a_tp^t, \qquad 0\leq a_i \leq p - 1
\end{equation*}
náttúrulega framsetningu þess. Þá er
\begin{equation*}
\mbox{ord}_p(n!) = \frac{n - s_n}{p-1}
\end{equation*}
þar sem $s_n = a_0 + a_1 + \ldots + a_t$.
\end{hjalparsetn}
%Örugglega fínt að laga þetta.
\begin{proof}
Til að fá þægilega frammsetningu setjum við $s_0 = 0$. Látum $1\leq k \leq k$ og lát
\begin{equation*}
k = 0,0 0 0 \ldots 0 b_\nu b_{\nu + 1} \ldots b_t ,  
\end{equation*}
vera náttúrulega frammsetningu $k$ þar sem $b_\nu \geq 1$, þ.e.a.s. 
\begin{equation*}
k = b_\nu p^\nu + b_{\nu +1} p^{\nu+1} + \ldots + b_t p^t, \qquad s_k = b_\nu + b_{\nu+1} + \ldots + b_t.
\end{equation*}
Þá er 
\begin{align*}
k-1 = (p-1) & + (p-1)p + \ldots + (p-1)p^{\nu +1} &\\
 & + (b_\nu -1)p^\nu + b_{\nu+1} p^{\nu +1} + \ldots + b_t p^t &
\end{align*}
og 
\begin{align*}
s_{k-1} &= \nu(p-1) + (b_\nu -1) + b_{\nu+1} + \ldots + b_t & \\
 &  = \nu(p-1) + s_k -1 &
\end{align*}
Því fæst
\begin{equation*}
\nu = \frac{s_{k-1}-s_k + 1}{p-1 },
\end{equation*}
en klárlega er $\nu = \mbox{ord}_p k.$ , svo að
\begin{equation*}
\mbox{ord}_p k = \frac{s_{k-1 - s_k +1}}{p-1}
\end{equation*}
og nú fáum við með því að nota \ref{starwars} að 
\begin{equation*}
\mbox{ord}_p(n!) = \frac{1}{p-1} \sum_{k=1}^n (s_{k-1} - s_k +1) = \frac{n - s_n}{p-1}.
\end{equation*}
\end{proof}

\begin{setn}
Samleitnisvæðið fyrir veldvísisröðina $E(x) = \sum_{n=0}^{\infty} x^n/n!$ er mengi allra x þ.a.
$\mbox{ord}_px>1/(p-1)$.
\end{setn}
\begin{proof}
Með því að nota \ref{starwars} og hjálparsetninguna hér að ofan fáum við
\begin{align*}
\mbox{ord}_p \left( \frac{x^n}{n!} \right) & =  n \mbox{ord}_p x - \mbox{ord}_p(n!) & \\
 & =  n \mbox{ord}_p x - \left( \frac{n - s_n}{p-1} \right) &\\ 
 & =  n \left( \mbox{ord}_p x - \frac{1}{p-1} \right) + \frac{s_n}{p-1}&
\end{align*}
svo ef $\mbox{ord}_p x > 1/(p-1)$ þá $\lim_{n\rightarrow \infty } \mbox{ord}_p (x^n/n!) \rightarrow \infty$ 
og því $\|x^n/n!\|_p \rightarrow 0$, þ.e.a.s röðin er samleitin.
Ef aftur á móti $\mbox{ord}_px \leq 1/(p-1)$, þá fæst að fyrir öll $n$ af gerðinni $p^k$ að $s_n = 1$ og því $\mbox{ord}_p(x^n/n!) \nrightarrow \infty$.
\end{proof}
\begin{daemi}
%útskýra
Ef $p=2$ og $x \in \mathbb{Z}$, þá gefur seinasta setning af sér að $x$ verður að vera margfeldi af 4 til að vera í samleitnigeisla veldisvísisfallsins. 
\end{daemi}

\begin{setn}
Samleitnisvæðið fyrir lograröðina $\log(1+x) = \sum_{n=1}^{\infty} (-1)^{n-1}x^n/n$, er mengi allra x þ.a. $\mbox{ord}_px>0$.
\end{setn}
\begin{proof}
Við höfum að 
\begin{equation} \label{daft}
\mbox{ord}_p \left( \frac{(-1)^{n-1}x^n}{n} \right) = n \mbox{ord}_px - \mbox{ord}_p n.
\end{equation}
Hinsvegar vitum við að 
\begin{equation*}
p^{\mbox{ord}_p n} \leq n,
\end{equation*}
og því 
\begin{equation*}
\mbox{ord}_p n \leq \frac{\log n}{\log p}.
\end{equation*}
Með því að stinga þessu inní jöfnu \ref{daft} sýnir að ef $\mbox{ord}_p x> 0$, þá
\begin{equation*}
\mbox{ord}_p \left( \frac{(-1)^{n-1}x^n}{n} \right) \rightarrow \infty,
\end{equation*}
og röðin er því samleitin. Ef aftur á móti $\mbox{ord}_p x \leq 0$, þá fæst að fyrir þau n þ.a. $p \nmid n $ að $\mbox{ord}_pn=0$, og
\begin{equation*}
\mbox{ord}_p \left( \frac{(-1)^{n-1}x^n}{n} \right) \nrightarrow \infty
\end{equation*}
\end{proof}

\begin{setn}
Lát $y\in \mathbb{Q}_p$ vera þannig að $\mbox{ord}_p y \geq 0$. Tvíliðunarröðin 
\begin{equation*}
(1+x)^y = \sum_{n=0}^\infty \binom{y}{n}x^n
\end{equation*}
er samleitin fyrir öll $x$ með $\mbox{ord}_px>1/(p-1)$.
\end{setn}
\begin{proof}
Fyrst að $\mbox{ord}_py\geq0$ og 
\begin{equation*}
\binom{y}{n} = \frac{y(y-1)\ldots (y-(n-1))}{n!},
\end{equation*}
þá er ljóst að
\begin{equation*}
\mbox{ord}_p \binom{y}{n} \geq \mbox{ord}_p \left( \frac{1}{n!} \right).
\end{equation*}
Því fæst
\begin{equation*}
\mbox{ord}_p \left( \binom{y}{n} x^n \right) \geq \mbox{ord}_p \frac{(x^n) }{n!},
\end{equation*}
og við sýndum í sönnuninni á samleitnisvæðis veldívísisfallsins að seinasta jafnan stefnir á óendalegt ef að $x$ er innan samleitnisvæðisins.
\end{proof}


%bls 47 Mahler.
\section*{Samfelld föll}
Látum $I=[Q_p]$ tákna mengi allra p-legra heiltalna
\begin{equation*}
x = a_0 + a_1p+a_2p^2 + \ldots \qquad (a_k \mbox{digits}), 
\end{equation*} 
þ.e.a.s. mengi p-legra talna sem uppfylla
\begin{equation*}
 |x|_p \leq 1 .
\end{equation*}
%Setja skilgreiningar haus hér.
Við skilgreinum föll
%Laga þessa setningu eitthvað. 
\begin{equation*}
f:I \rightarrow Q_p
\end{equation*}
samfelld ef fyrir 
\begin{equation*}
x_0 = \lim_{n \rightarrow \infty} x_n
\end{equation*}
gildir
\begin{equation*}
f(x_0) = \lim_{n \rightarrow \infty} f(x_n).
\end{equation*}
Jafngilt er að segja að fall er samfellt e.f.f. fyrir sérhverja heiltölu $s\geq 0$ er til heiltala $t = t(x_0,s)\geq 0$ þ.a.
\begin{equation*}
\mbox{ef } |x-x_0|_p \leq p^{-t},\qquad \mbox{  þá } |f(x)-f(x_0)|_p \leq p^{-s}.
\end{equation*}
Einnig má yfirfæra skilgreininguna á almennari bil
\begin{equation*}
I^* : |x-a|_p \leq p^{-r}
\end{equation*} 
þar sem $a \in I$ er fast og r er fest ræð tala; en þá gefur línulega vörpunin 
\begin{equation*}
x \rightarrow p^rx+a
\end{equation*}
fyrri skilgreiningu.
%Setning
\begin{setn}
Ef $f(x)$ er samfellt á $I$, þá er $f(x)$ samfellt í jöfnum mæli á I, þ.e.a.s fyrir hvert $s$, er til $t=t(s)$ óháð $x_0$ þ.a 
\begin{equation}
\mbox{ef   } x,x_0\in I \mbox{ og  } |x-x_0|_p \leq p^{-t}, \qquad \mbox{þá  } |f(x)-f(x_0)|_p \leq p^{-s}.
\end{equation}
\end{setn} 
\end{document}
