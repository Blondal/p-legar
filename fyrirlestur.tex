\documentclass[12pt]{article}
\usepackage[icelandic,english]{babel}
\usepackage{a4,t1enc,html,url}
\usepackage{ucs}
\usepackage{times}
\usepackage[utf8x]{inputenc}
%\pagestyle{empty}
\usepackage{graphicx,latexsym}
\parindent 0pt
%
\usepackage{amsmath} 
\usepackage{graphicx,latexsym} 
\usepackage{amssymb}
\author{Máni Maríus Viðarsson, Jón Blöndal}
%----------------------------------------------------------



\title{p-Legar tölur}
\begin{document}
\maketitle
\section*{Fyrstu kynni - dæmi}
%Hér setjum við sýnidæmi til að vekja áhuga.
Í lok þessara tveggja fyrirlestra mun ykkur vonandi ekki brenna í augun við að sjá eftirfarandi jöfnu:
\begin{equation*}
-1 = 4 + 4\cdot 5+4\cdot 5^2+4\cdot 5^3 \ldots
\end{equation*}
%eða 
%\begin{equation*}
%\sqrt{7} = 1 + 1\cdot 3 + 1\cdot 3^2+0\cdot 3^3+2\cdot 3^4 \ldots
%\end{equation*}

%Hér kemur fyrri fyrirlestur
\section*{Inngangur}
Festum prímtölu p og skoðum formlegu veldaröðina
\begin{align*}
a = a_0 + a_1 p + a_2 p^2 + ...,
\end{align*}
þar sem $0 \leq a_i \leq p-1$ er heiltala. 
Svona veldaraðir er hægt að leggja saman, draga frá hvorri annarri
 og margfalda.
Með þessum hætti fáum við bauginn $\mathbb{Z}_p$, p-legu heiltölurnar.
Þær hafa núllstakið $0 = 0+0p+0p^2+...$ og 
einingarstakið $1 = 1 + 0p+0p^2+...$.
\\Baugurinn $\mathbb{Z}_p$ hefur enga núlldeila svo hann er heilbaugur 
og hefur brotasviðið $\mathbb{Q}_p$ sem kallað er svið p-legu talnanna.

%Hér viljum við skilgreina allt reglubundið
\section*{Skilgreiningar}
Ef við skerum af formlegu veldaröðinni við einhvert veldið á p fæst heil tala 
\begin{align*}
a = a_0 + a_1 p + a_2 p^2 + \ldots + a_{n-1}p^{n-1}. 
\end{align*}
Munum að í baugnun $\mathbb{Z}_p$ má geyma og fá lánað í samlagningu, ef við 
fáum $1$ lánaðan frá $p^n$ og skerum af við veldið $n$, fáum við 
\begin{align*}
 a_0 + a_1 p + a_2 p^2 + \ldots + (a_{n-1} + p)p^{n-1}.
\end{align*}
Þessi afskurður verður aðeins vel skilgreindur ef við túlkum hvert 
stak eftir afskurðinn sem stak í $\mathbb{Z}/p^n \mathbb{Z}$, þ.e. það 
er í lagi að geyma og fá lánað.\\
\\Skoðum nú vörpunina
\begin{align*}
 \centering \phi_n : \mathbb{Z}/p^{n+1} \mathbb{Z} \rightarrow 
 \mathbb{Z}/p^{n} \mathbb{Z}, \ \  \left[\sum_{j=0}^k a_j p^j \right]
_{p^{k+1}} \mapsto \left[\sum_{j=0}^{k-1} a_j p^j \right]_{p^k}.
\end{align*}
Við fáum
\begin{align}\label{Zruna}
 \ldots \rightarrow \mathbb{Z}/p^{3} \mathbb{Z} \rightarrow \mathbb{Z}/p^{2} 
\mathbb{Z} \rightarrow \mathbb{Z}/p \mathbb{Z}
\end{align}
\begin{skilgr}
 Baugur p-legu heiltalnanna $\mathbb{Z}_p$ er öfuga markgildið af 
röðinni $\ref{Zruna}$
\end{skilgr}
Stak $x$ í $\mathbb{Z}_p$ er runan $(\ldots , x_n, \ldots, x_1)$ þar sem 
$x_n \in \mathbb{Z}/p^n\mathbb{Z}$ og $\phi_n(x_{n+1}) = x_{n}$.
Athugum nú að $\mathbb{Z}_p$ er hlutbaugur í mengjamargfeldinu
 $\prod_{n \geq 1} \mathbb{Z}/p^n\mathbb{Z}$. Ef við setjum á 
$\mathbb{Z}/p^n\mathbb{Z}$ strjála grannmynstrið og á 
$\prod_{n \geq 1} \mathbb{Z}/p^n\mathbb{Z}$ 
margfeldisgrannmynstrið, erfir $\mathbb{Z}_p$ grannmynstrið sem 
gerir það að þjöppuðu rúmi.\\
%\\Látum nú $\varepsilon_n : \mathbb{Z}_p \rightarrow \mathbb{Z}/p^n 
%\mathbb{Z}$ vera fallið sem varpar p-legri heiltölu $x$ á n-ta stakið $x_n$.
%\begin{setn}
% Lestin $0 \rightarrow \mathbb{Z}_p \rightarrow^{p^n} \mathbb{Z}_p 
%\rightarrow^{\varepsilon_n} \mathbb{Z}/p^n\mathbb{Z} \rightarrow 0$ 
%er fleyguð.
%\end{setn}
%(Því getum við greint $\mathbb{Z}_p/p^n\mathbb{Z}$ með 
%$\mathbb{Z}/p^n\mathbb{Z}$).
% \begin{proof}
% Margföldun með $p$ (og því einnig með $p^n$) er eintæk í 
% $\mathbb{Z}_p$; ef $x = (x_n)$ er p-leg heiltala þ.a. $px = 0$,
% fáum við $px_{n+1} = 0$ fyrir öll $n$, og $x_{n+1}$ er á forminu 
% $p^ny_{n+1}$ með $y_{n+1} \in \mathbb{Z}_p/p^{n+1}\mathbb{Z}$; 
% fyrst $x_n = \phi_{n+1}(x_{n+1})$, sjáum við að $x_n$ er deilanleg með 
% $p^n$ svo það er núll.\\
% Það er skýrt að $p^n \mathbb{Z}_p \subset Ker(\varepsilon_n)$; öfugt ef 
% $x = (x_m)$ er í 
% \end{proof}
\begin{setn}
 (a) Til að stak í $\mathbb{Z}_p$ hafi umhverfu er nauðsynlegt og nægjanlegt 
að það sé ekki deilanlegt með $p$.\\
 (b) Ef $\mathbb{U}$ er mengi umhverfanlegra staka í $\mathbb{Z}_p$ er hægt að 
skrifa hvert stak (nema núll) á óvtvírætt ákvarðaðan hátt á forminu $p^nu$ þar
 sem $u \in \mathbb{U}$ og $n\geq 0$.
\end{setn}
Látum nú $\varepsilon_n : \mathbb{Z}_p \rightarrow \mathbb{Z}/p^n 
\mathbb{Z}$ vera fallið sem varpar p-legri heiltölu $x$ á n-ta stakið $x_n$.
\begin{proof}
 Nóg að sanna (a) fyrir $\mathbb{Z}/p^n\mathbb{Z}$, tilfellið fyrir 
$\mathbb{Z}_p$ fylgir. Ef $x \in \mathbb{Z}/p^n\mathbb{Z}$ tilheyrir 
ekki $p\mathbb{Z}/p^n\mathbb{Z}$, þá er mynd þess í 
$\mathbb{Z}/p\mathbb{Z}$ ekki núll, og því umhverfanlegt: 
því eru til $y,z \in \mathbb{Z}/p^n\mathbb{Z}$ þannig að 
$xy = 1-pz$, og því 
\begin{align*}
 xy(1+pz+\ldots+p^{n-1}z^{n-1}) = 1,
\end{align*}
sem sannar að x er umhverfanlegt.\\
Hinsvegar, ef $x \in \mathbb{Z}_p$ er ekki núll, þá er til stærsta heiltala 
$n$ þannig að $x_n = \varepsilon_n(x)$ er núll; þá er $x = p^nu$ með $u$ ekki 
deilanlegt með p, og því er $u \in \mathbb{U}$ vegna (a). 
\end{proof}
\begin{skilgr}
 Virðing á sviði K er vörpun $K \rightarrow \mathbb{R}$ þ.a.
\begin{align*}
 i)& \  |x|\geq 0; |x| = 0 \mbox{ þþaa } x = 0, \\
 ii)& \  |xy| = |x||y|, \\
 iii)& \  |x+y| \leq |x| + |y|.
\end{align*}
Við segjum að virðing sé óarkímedísk ef $|x+y| \leq \max \{ |x|,|y| \}$.
\end{skilgr}
\begin{Rit}
Látum $x \in \mathbb{Z}_p$, $x \neq 0$; skrifum $x$ á forminu $p^nu$ með 
$u \in \mathbb{U}$. Talan $n$ er kölluð $p$-lega virðingin á $x$ og er skrifuð sem 
$v_p(x)$. Við setjum $v_p(0) = +\infty$ og höfum 
\begin{align*}
 v_p(xy) &= v_p(x) + v_p(y), \\
 v_p(x+y) &\geq \inf(v_p(x),v_p(y)).
\end{align*}
Augljóst er af þessum formúlum að $\mathbb{Z}_p$ er heilbaugur.
\end{Rit}
\begin{setn}
 Grannmynstrið á $\mathbb{Z}_p$ má skilgreina með firðinni
\begin{align}
 d(x,y) = e^{-v_p(x-y)}.
\end{align}
Baugurinn $\mathbb{Z}_p$ er fullkomið firðrúm og $\mathbb{Z}$  er þétt 
í honum.
\end{setn}
\begin{proof}
 Íðulin $p^n \mathbb{Z}_p$ mynda grunn af grenndum um 0 og vegna þess að 
$x \in p^n \mathbb{Z}_p$ er jafngilt því að $v_p(x) \geq n$ er grannmynstrið 
á $\mathbb{Z}_p$ skilgreint með firðinni $d(x,y) = e^{-v_p(x-y)}$. 
Fyrst að $\mathbb{Z}_p$ er þjappað, er það fullkomið. Að lokum, ef $x = (x_n)$
 er stak í $\mathbb{Z}_p$ og ef $y_n \in \mathbb{Z}$ þannig að 
$y_n \equiv x_n(\mbox{mod} p^n)$, þá er $\lim y_n = x$, sem sannar að $\mathbb{Z}$ 
er þétt í $\mathbb{Z}_p$.
\end{proof}

\begin{skilgr}
 Svið $p$-legu talnanna $\mathbb{Q}_p$ er brotasvið baugsins $\mathbb{Z}_p$.
\end{skilgr}
Við sjáum strax að $\mathbb{Q}_p = \mathbb{Z}_p[p^{-1}]$. Hvert stak 
$x \in \mathbb{Q}_p^*$ er hægt að skrifa á ótvírætt ákvarðaðan hátt á forminu 
$p^nu$ með $n \in \mathbb{Z}$, $u \in \mathbb{U}$; hér er $n$ aftur $p$-lega 
virðingin af $x$ og er skrifað sem $v_p(x)$. Við höfum $v_p(x) \geq 0$ þþaa 
$x \in \mathbb{Z}_p$. 
\begin{setn}
 Sviðið $\mathbb{Q}_p$, með grannmynstrinu skilgreindu af 
$d(x,y) = e^{-v_p(x-y)}$, er staðþjappað og inniheldur $\mathbb{Z}_p$ sem 
hlutbaug; sviðið $\mathbb{Q}$ er þétt í $\mathbb{Q}_p$.
\end{setn}
\begin{proof}
 Augljóst.
\end{proof}
$Athugasemdir$\\
(1) Við hefðum getað skilgreint $\mathbb{Q}_p$ sem fullkomnuna á $\mathbb{Q}$ 
fyrir $p$-legu firðina $d$.\\
(2) Firðin $d$ er ofurfirð, það er 
\begin{align*}
 d(x,y) \leq \sup (d(x,y),d(y,z)).
\end{align*}
Af þessu sjáum við að runa $u_n$ hefur markgildi þþaa 
\begin{align*}
 \lim(u_{n+1} - u_n) = 0;
\end{align*}
og á sama hátt, röð er samleitin þþaa runa hennar stefni á 0.\\
\\Við fáum að hvert stak í $\mathbb{Q}_p$ má nú skrifa sem summu
\begin{align*}
 \sum_{j=N}^{\infty} a_jp^j,\ \ N \in \mathbb{Z}
\end{align*}
\begin{Rit}
Ritum $p$-lega tölu $x$ nú á forminu
\begin{align*}
 \ldots x_2 x_1,x_0 \ldots x_{-v} (p)
\end{align*}
\end{Rit}
\begin{daemi}
Reiknum smá \\
Samlagning í $\mathbb{Q}_7$:\\
\begin{align*}
1\ 11\ 11\ &1 \\
 216731,&254 \\
+\ \underline{2513125},&\underline{73} \\
3033160,&314
\end{align*}
Margföldun í $\mathbb{Q}_7$:\\
\begin{align*}
 3,&21 \\
\cdot \ \underline{2},&\underline{1 \ \ } \\
3\ &21 \\
+ \ \underline{64\ }&\underline{1 \ \ } \\
100,&31
\end{align*}
%Deiling í $\mathbb{Q}_5$:\\
%\begin{align*}
%21,43|31,23& = \overline{034430},2 \\
%43\ 23
%\end{align*}

\end{daemi}

\section*{Eiginleikar ofurfirða}
\begin{skilgr}
Smá upprifjun. Látum
\begin{align*}
B(a,r) = \{x:d(x,a)<r\} \\
S(a,r) = \{x:d(x,a) = r\} \\
B^+(a,r) = \{x:d(x,a) \leq r\}
\end{align*}
tákna opnu kúluna, kúluhvelið og lokuðu kúluna með geisla $r$ um punktinn $a$.
\end{skilgr}
Þar sem við erum að skoða ofurfirðir þá gefur sterka þríhyrnings-ójafnan af sér nokkra skemmtilega eiginleika:
\begin{setn}
Ef $b \in S(a,r)$ þá er $B(b,r) \subset S(a,r)$.
\end{setn}
\begin{proof}
Tökum $c \in B(b,r)$. Þá er fjarlægðin milli $a$ og $c$
\begin{equation*}
d(a,c) \leq \max(d(a,b),d(b,c)) = r 
\end{equation*}
svo $c \in S(a,r)$.
\end{proof}
Þessi setning gefur af sér 2 fylgisetnigar:
\begin{fylgisetn}
Allar lokaðar kúlur eru opin mengi og allar opnar kúlur eru lokuð mengi.
\end{fylgisetn}
\begin{fylgisetn} 
Ofurfirðrúm er algjörlega ósamanhangandi; einu samanhangandi mengin eru einstökungar.
\end{fylgisetn}
Einn áhugaverðan eiginlega getum við bent á að lokum:
\begin{setn}
Ef $b\in B(a,r)$ þá er $B(a,r) = B(b,r)$. M.ö.o. sérhver punktur í kúlu er miðpunktur hennar.
\end{setn}
\begin{proof}
Tökum $c\in B(b,r)$. Þá fæst
\begin{equation*}
d(a,c) = \max (d(a,b),d(b,c)) \leq r
\end{equation*}
svo $B(b,r) \subset B(a,r)$. Eins fæst að $B(a,r) \subset B(b,r)$, svo að $B(a,r) = B(b,r)$.
\end{proof}
%Hér byrjar svo seinni fyrirlestur


%bls 25 Bachmann
\section*{Samleitnar runur}
\begin{setn}
Ef $k$ er fullkomið svið m.t.t. óarkímedískar virðingar $\| \cdot  \|$, og ef $\{ a_n \} $ er 
runa af stökum í $k$ þ.a. $\lim a_n = 0$, þá er $\sum_{n=1}^\infty a_n$ samleitin.
\end{setn}
\begin{proof}
Lát 
\begin{equation*}
s_n = a_1+ \ldots + a_n
\end{equation*}
og
\begin{equation*}
s_m = a_1 + \ldots + a_m,
\end{equation*}
þar sem $m < n$. Þá
\begin{equation*}
\|s_n - s_m \| = \| a_n + \ldots + a_{m+1} \| \leq \max_{m+1 \leq i \leq n} \|a_i\| \rightarrow 0
\end{equation*}
þegar $n,m \rightarrow \infty$, því er $\lim s_n$ til þar sem $k$ er fullkomið. Þ.a.l er $\sum_{n=1}^\infty a_n$ samleitin.
\end{proof}

\begin{setn}
Ef $k$ er fullkomið svið m.t.t. virðingar $\| \cdot \|$, og ef $\{ a_n \}$ er runa af stökum í $k$ 
þ.a. $\sum_{n=1}^{\infty} \| a_n \| $ er samleitin, þá er $\sum_{n=1}^\infty a_n $ samleitin. 
\end{setn}
\begin{proof}
Lát 
\begin{equation*}
 s_n' = \| a_1 \| + \ldots + \|a_n\|, \qquad s_n = a_1 + \ldots + a_n.
\end{equation*}
Þá fyrir $n>m$, 
\begin{equation*}
|s_n'- s_m' | = | \| a_n \| + \ldots \|a_{m+1}\| |.
\end{equation*}
þar sem algildið $| s_n' - s_m' | $ vísir til venjulega  tölugildisins á $\mathbb{R}$. En 
\begin{equation*}
| \| a_n \| + \ldots + \| a_{m+1} \| | = \| a_n \| + \ldots + \| a_{m+1} \|
\end{equation*}
því $\| a_i \| \leq 0$, og 
\begin{equation}
\| s_n - s_m \| = \|a_n\| + \ldots + \|a_{m+1}\| \rightarrow 0
\end{equation}
þegar $n,m \rightarrow \infty$ fyrst að $\sum_{n=1}^\infty \|a_n\|$ er samleitin. 
Því fæst $\|s_n - s_m \| \rightarrow 0$ þegar $n,m \rightarrow \infty $ og þar sem $k$ er fullkomið fæst niðurstaðan.  
\end{proof}

Fyrir veldaröð $\sum_{n=0}^{\infty} a_nx^n$ í $\mathbb{R}$ eða $\mathbb{C}$ segjum við að hún hafi samleitnigeislann $r$ 
m.t.t. p-lega staðalsins ef fyrir öll x þ.a. $|x|_p < r$ veldaröðin er samleitin, en ósamleitin fyrir öll x þ.a. $|x|_p > r$, 
þar sem r er gefið með
\begin{equation*}
r = \frac{1}{\limsup \sqrt{|a_n|_p }}.
\end{equation*} 


%----
\begin{hjalparsetn}
Lát $n$ vera jákvæða heiltölu og 
\begin{equation*}
n = a_0 + a_1p+ \ldots + a_tp^t, \qquad 0\leq a_i \leq p - 1
\end{equation*}
náttúrulega framsetningu þess. Þá er
\begin{equation*}
v_p(n!) = \frac{n - s_n}{p-1}
\end{equation*}
þar sem $s_n = a_0 + a_1 + \ldots + a_t$.
\end{hjalparsetn}
%Örugglega fínt að laga þetta.
\begin{proof}
Til að fá þægilega framsetningu setjum við $s_0 = 0$. Látum $1\leq k \leq 0$ og lát
\begin{equation*}
k = 0,0 0 0 \ldots 0 b_\nu b_{\nu + 1} \ldots b_t ,  
\end{equation*}
vera náttúrulega framsetningu $k$ þar sem $b_\nu \geq 1$, þ.e.a.s. 
\begin{equation*}
k = b_\nu p^\nu + b_{\nu +1} p^{\nu+1} + \ldots + b_t p^t, \qquad s_k = b_\nu + b_{\nu+1} + \ldots + b_t.
\end{equation*}
Þá er 
\begin{align*}
k-1 = (p-1) & + (p-1)p + \ldots + (p-1)p^{\nu +1} &\\
 & + (b_\nu -1)p^\nu + b_{\nu+1} p^{\nu +1} + \ldots + b_t p^t &
\end{align*}
og 
\begin{align*}
s_{k-1} &= \nu(p-1) + (b_\nu -1) + b_{\nu+1} + \ldots + b_t & \\
 &  = \nu(p-1) + s_k -1 &
\end{align*}
Því fæst
\begin{equation*}
\nu = \frac{s_{k-1}-s_k + 1}{p-1 },
\end{equation*}
en klárlega er $\nu = v_p k.$ , svo að
\begin{equation*}
v_p k = \frac{s_{k-1 - s_k +1}}{p-1}
\end{equation*}
og nú fáum við með því að nota \ref{starwars} að 
\begin{equation*}
v_p(n!) = \frac{1}{p-1} \sum_{k=1}^n (s_{k-1} - s_k +1) = \frac{n - s_n}{p-1}.
\end{equation*}
\end{proof}

\begin{setn}
Samleitnisvæðið fyrir veldvísisröðina $E(x) = \sum_{n=0}^{\infty} x^n/n!$ er mengi allra x þ.a.
$v_p(x)>1/(p-1)$  $(\iff |x|_p < p^{- \frac{1}{p-1}} )$.
\end{setn}
\begin{proof}
Með því að nota \ref{starwars} og hjálparsetninguna hér að ofan fáum við
\begin{align*}
v_p \left( \frac{x^n}{n!} \right) & =  n v_p x - v_p(n!) & \\
 & =  n v_p x - \left( \frac{n - s_n}{p-1} \right) &\\ 
 & =  n \left( v_p x - \frac{1}{p-1} \right) + \frac{s_n}{p-1}&
\end{align*}
svo ef $v_p x > 1/(p-1)$ þá $\lim_{n\rightarrow \infty } v_p (x^n/n!) \rightarrow \infty$ 
og því $|x^n/n!|_p \rightarrow 0$, þ.e.a.s röðin er samleitin.
Ef aftur á móti $v_px \leq 1/(p-1)$, þá fæst að fyrir öll $n$ af gerðinni $p^k$ að $s_n = 1$ og því $v_p(x^n/n!) \nrightarrow \infty$.
\end{proof}

Sjáum nú að ef veldisvísisröðin er samleitin fyrir $x$ og $y$ þá gefur \ref{starwars} að hún er samleitin fyrir $x+y$. Því fæst
\begin{equation*}
E(x+y) = E(x) + E(y)
\end{equation*}

\begin{daemi}
%útskýra
Ef $p=2$ og $x \in \mathbb{Z}$, þá gefur seinasta setning af sér að $x$ verður að vera margfeldi af 4 til að vera í samleitnigeisla veldisvísisfallsins. 
\end{daemi}

\begin{setn}
Samleitnisvæðið fyrir lograröðina $\log(1+x) = \sum_{n=1}^{\infty} (-1)^{n-1}x^n/n$, er mengi allra x þ.a. $v_p(x)>0$  $(\iff |x|_p<1)$.
\end{setn}
\begin{proof}
Við höfum að 
\begin{equation} \label{daft}
v_p \left( \frac{(-1)^{n-1}x^n}{n} \right) = n v_px - v_p n.
\end{equation}
Hinsvegar vitum við að 
\begin{equation*}
p^{v_p n} \leq n,
\end{equation*}
og því 
\begin{equation*}
v_p n \leq \frac{\log n}{\log p}.
\end{equation*}
Með því að stinga þessu inní jöfnu \ref{daft} sýnir að ef $v_p x> 0$, þá
\begin{equation*}
v_p \left( \frac{(-1)^{n-1}x^n}{n} \right) \rightarrow \infty,
\end{equation*}
og röðin er því samleitin. Ef aftur á móti $v_p x \leq 0$, þá fæst að fyrir þau n þ.a. $p \nmid n $ að $v_pn=0$, og
\begin{equation*}
v_p \left( \frac{(-1)^{n-1}x^n}{n} \right) \nrightarrow \infty
\end{equation*}
\end{proof}

Eins og fyrir veldisvísisfallið fáum við að ef lograröðin er samleitin fyrir $x$ og $y$ þá er hún samleitin fyrir 
\begin{equation*}
\log (1+x)(1+y) = \log(1+x) + \log(1+y)
\end{equation*} 

\begin{setn}
Lát $y\in \mathbb{Q}_p$ vera þannig að $v_p y \geq 0$. Tvíliðunarröðin 
\begin{equation*}
(1+x)^y = \sum_{n=0}^\infty \binom{y}{n}x^n
\end{equation*}
er samleitin fyrir öll $x$ með $v_px>1/(p-1)$  $( \iff |x|_p <p^{-\frac{1}{p-1}})$.
\end{setn}
\begin{proof}
Fyrst að $v_py\geq0$ og 
\begin{equation*}
\binom{y}{n} = \frac{y(y-1)\ldots (y-(n-1))}{n!},
\end{equation*}
þá er ljóst að
\begin{equation*}
v_p \binom{y}{n} \geq v_p \left( \frac{1}{n!} \right).
\end{equation*}
Því fæst
\begin{equation*}
v_p \left( \binom{y}{n} x^n \right) \geq v_p \frac{(x^n) }{n!},
\end{equation*}
og við sýndum í sönnuninni á samleitnisvæðis veldívísisfallsins að seinasta jafnan stefnir á óendalegt ef að $x$ er innan samleitnisvæðisins.
\end{proof}

Með því að nota það sem við vitum nú getum við komist að því að fyrir $v_p(y) \geq 0$ og $v_p(x) > 1/(p-1)$ fæst að
\begin{equation*}
(1+x)^y = E(y \log(1+x)),
\end{equation*}  
og 
\begin{equation*}
\log(1+x)^y = y \log(1+x).
\end{equation*}
Sönnun á þessum jöfnum er þó sleppt hér.
%bls 47 Mahler.
\section*{Samfelld föll}

%Setja skilgreiningar haus hér.
\begin{skilgr}
Við skilgreinum föll
%Laga þessa setningu eitthvað. 
\begin{equation*}
f:\mathbb{Z}_p \rightarrow Q_p
\end{equation*}
samfelld ef fyrir 
\begin{equation*}
x_0 = \lim_{n \rightarrow \infty} x_n
\end{equation*}
gildir
\begin{equation*}
f(x_0) = \lim_{n \rightarrow \infty} f(x_n).
\end{equation*}
Jafngilt er að segja að fall er samfellt e.f.f. fyrir sérhverja heiltölu $s\geq 0$ er til heiltala $t = t(x_0,s)\geq 0$ þ.a.
\begin{equation*}
\mbox{ef } |x-x_0|_p \leq p^{-t},\qquad \mbox{  þá } |f(x)-f(x_0)|_p \leq p^{-s}.
\end{equation*}
Einnig má yfirfæra skilgreininguna á almennari bil
\begin{equation*}
I^* : |x-a|_p \leq p^{-r}
\end{equation*} 
þar sem $a \in I$ er fast og r er fest ræð tala; en þá gefur línulega vörpunin 
\begin{equation*}
x \rightarrow p^rx+a
\end{equation*}
fyrri skilgreiningu.
\end{skilgr}

\begin{hjalparsetn}
Sérhver óendanleg röð af p-legum heiltölum inniheldur samleitna óendanlega hlutrunu. 
\end{hjalparsetn}
\begin{proof}
Öll stök $x_n$ í runu $S$, eru af gerðinni
\begin{equation*}
x_n = a_{0n} + a_{1n}p+a_{2n}p^2 + \ldots \qquad (n= 1,2,3,\ldots)
\end{equation*}
þar sem a-in eru heiltölur sem taka mest endanlega mörg gildi $0,1, \ldots p-1$.
Því getum við smíðað óendanlega hlutrunu $S_0$ af þar sem öll $a_0n$-in hafa fast gildi $a_0$, svo getum við 
smíðað óendanlega hlutrunu $S_1$ í $S_0$ með því að festa $a_1n$ og koll af kolli.
Látum $x^{(n)}$ tákna n-ta stakið í $S_n$, þá er $\{x^{(1)},x^{(2)},\ldots \}$ óendanleg hlutruna af $S$ sem stefnir á $x$. 
\end{proof}


\begin{setn}
Ef $f(x)$ er samfellt á $\mathbb{Z}$, þá er $f(x)$ samfellt í jöfnum mæli á I, þ.e.a.s fyrir hvert $s$, er til $t=t(s)$ óháð $x_0$ þ.a 
\begin{equation}
\mbox{ef   } x,x_0\in \mathbb{Z} \mbox{ og  } |x-x_0|_p \leq p^{-t}, \qquad \mbox{þá  } |f(x)-f(x_0)|_p \leq p^{-s}.
\end{equation}
\end{setn} 
\begin{proof}
G.r.f. að fullyrðingin sé röng. Fyrir gefið $s$, sama hversu stórt $t \geq 0$ er valið, eru þá til p-legar heiltölur $x_0$ og $x$ þ.a.
\begin{equation*}
|x_0-x|_p \leq p^{-t},
\end{equation*}
en
\begin{equation*}
|f(x_0)-f(x)|_p > p^{-s}.
\end{equation*}
Því megum við velja tvær runur af p-legum heiltölum þ.a.
\begin{equation*}
\{x_0^{(1)}, x_0^{(2)}, \ldots \} \qquad \mbox{og} \qquad \{x^{(1)}, x^{(2)}, \ldots\}
\end{equation*}
þannig að
\begin{equation*}
|x_0^{(n)}-x^{(n)} |_p \geq p^{-n}
\end{equation*}
en 
\begin{equation*}
|f(x_0^{(n)}) - f(x^{(n)})|_p > p^{-s}.
\end{equation*}
Ef $\{x_0^{(n)}\}$ er skipt út fyrir hvaða óendanlegu hlutrunu sem er, gilda enn jöfnurnar hér að ofan. 
Nú gefur hjálparsetningin að markgildið
\begin{equation*}
\lim_{n\rightarrow \infty} x_0^{(n)}  = x_0
\end{equation*}
sé til og er p-leg heiltala. En einnig fæst að
\begin{equation*}
\lim_{n\rightarrow \infty} x^{(n)} = x_0
\end{equation*}
 fyrst að $\{x_0^{(n)} - x^{(n)} \}$ er núllruna. Því er 
\begin{equation*}
|x_0-x^{(n)}|_p = |(x_0 - x_0^{(n)} + (x_0^{(n)} -x^{(n)}) |_p \rightarrow 0,
\end{equation*}
svo að samfelldnin gefur nú
\begin{equation*}
\lim_{n\rightarrow \infty} f(x^{(n)}) = f(x_0).
\end{equation*}
En þetta er ómögulegt því
\begin{equation*}
|f(x_0)-f(x^{(n)})|_p = \lim_{n\rightarrow \infty} |f(x_0^{(n)}) - f(x^{(n)})|_p
\end{equation*}
er annaðhvort ekki til eða stærra en $p^{-s}$. Fráleitt!
\end{proof}

\begin{setn}
Ef $f(x)$ er samfellt fyrir öll $x\in \mathbb{Z}$ , þá er $f(x)$ takmarkað, þ.e.a.s. til er heiltala $m$ þ.a. 
\begin{equation*}
|f(x) |_p \leq p^m \qquad \mbox{fyrir öll   } x\in I.
\end{equation*}
\end{setn}
\begin{proof}
G.r.f. að staðhæfingin sé ósönn. Þá, fyrir hvern lið $m=1,2,3,\ldots,$ er til p-leg heiltala $x_m$ þ.a.
\begin{equation*}
|f(x_m)|_p > p^m.
\end{equation*}
Þar sem $I$ er þjappað megum við skipta $\{x_m\}$ út fyrir samleitna hlutrunu , því megum við g.r.f að 
\begin{equation*}
\lim_{n \rightarrow \infty} x_m = x_0
\end{equation*}
sé núþegar til og því p-leg heiltala. Fyrst að $f(x)$ er samfellt í $x_0$,
\begin{equation*}
f(x_0) = \lim_{m\rightarrow \infty} f(x_m),
\end{equation*}
og því 
\begin{equation*}
|f(x_0)|_p = \lim_{m \rightarrow \infty} |f(x_m)|_p = \infty,
\end{equation*}
sem er fráleitt!
\end{proof}

\begin{setn}
Sérhvert samfellt fall $f:\mathbb{Z} \rightarrow \mathbb{Q}$ má skrifa með nákvæmlega einum hætti sem
\begin{equation*}
f(x) = \sum_{n=0}^{+\infty} a_n \binom{x}{n}
\end{equation*} 
\end{setn}
\newpage
Hér kemur svo heimildaskráin.
\end{document}
